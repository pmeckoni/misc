\documentclass[10pt]{beamer}
\usetheme{Warsaw}
\usepackage[utf8]{inputenc}
\usepackage{amsmath}
\usepackage{amsfonts}
\usepackage{amssymb}
\usepackage{graphicx}
\usepackage{helvet}
\usepackage{xcolor}
\usepackage{hyperref}
\hypersetup{colorlinks,urlcolor=blue}
\usecolortheme{crane}
\author[Prashant Meckoni]{Prashant Meckoni}
\title[MATLAB Resources]{MATLAB Resources for Engineers}
\subtitle[v 0.1]{v 0.1}
%\setbeamercovered{transparent} 
%\setbeamertemplate{navigation symbols}{} 
%\logo{} 
\institute[UMass Amherst]{Mechanical \& Industrial Engineering, \\University of Massachusetts, Amherst } 
\date{February 25, 2016}
%\subject{MIE 397} 
\begin{document}

\begin{frame}
\titlepage
\end{frame}
\begin{frame}{Why MATLAB?}
\begin{itemize}
\item \href{http://stackoverflow.com/q/179904}{\textbf{``What is MATLAB good for? Why is it so used by universities? When is it better than Python?"} \\(or Java or C or Fortran)}\\ 
In my opinion, the best answer is available at\\ \url{http://stackoverflow.com/a/8347327}
\end{itemize}
\end{frame}
\begin{frame}{Getting Started}
\begin{itemize}
\item MATLAB is easy to learn. If you are new to MATLAB, you can probably start using it within an hour.
\item MATLAB is very flexible. If you are new to programming, you may not recognize its advantage unless you start using some other programming language or tool for similar tasks.
\item Many beginners stop using MATLAB as soon as their task, homework or project is completed and then struggle to use it again in another course. To become an expert, practice MATLAB on weekends and holidays with the intent to gain mastery over it.
\end{itemize}
\end{frame}
\begin{frame}{Software}
\begin{itemize}
\item MATLAB Subscription available to UMass students at No Cost: Sept 2015: Windows, Mac, Linux \url{http://www.it.umass.edu/news/2015-09-02/matlab-subscription-now-available-students-no-cost} 
\item University computers have MATLAB installed.
\item MATLAB can be used from the Virtual Lab remotely from your own computer. Windows, Mac, Linux\\ \url{https://engineering.umass.edu/about-us/engineering-computer-services/virtual-lab}
\item GNU Octave is free alternative to MATLAB with exactly same syntax to allow portability. Windows, Mac, Linux \url{http://www.gnu.org/software/octave/} 
\end{itemize}
\end{frame}
\begin{frame}{Tutorials}
\begin{itemize}
\item MATLAB Tutorials from MATLAB's website. Includes interactive tutorial if you have a Active license or Trial license  \url{http://www.mathworks.com/support/learn-with-matlab-tutorials.html}
\item ``Introduction to MATLAB For Engineering Students" by David Houcque 52 page compact introduction\\
 \url{http://www.mccormick.northwestern.edu/documents/students/undergraduate/introduction-to-matlab.pdf}
 \item ``Basic Concepts in Matlab" by Michael G. Kay 16 page introduction\\
 \url{http://www.ise.ncsu.edu/kay/Basic_Concepts_in_Matlab.pdf}
\item MIT OpenCourseWare ``Introduction to MATLAB Programming". You can learn using video lessons or using text and notes \url{http://ocw.mit.edu/courses/mathematics/18-s997-introduction-to-matlab-programming-fall-2011/}
\end{itemize}
\end{frame}
\begin{frame}{Books}
If you prefer a book, then try from the library
\begin{itemize}
\item ``Essential MATLAB for engineers and scientists" by Brian D Hahn; Daniel T Valentine available as print book and eBook \\
\url{http://umass.worldcat.org/oclc/806198305}

\item ``MATLAB primer by Timothy A Davis" available as print book and eBook \\ 
\url{http://umass.worldcat.org/oclc/645789683}

\end{itemize}
\end{frame}
\begin{frame}{Tips for Advanced users}
\begin{itemize}
\item ``MATLAB Style Guidelines 2.0" by Richard Johnson. Described as "A guide for MATLAB programmers that offers a collection of standards and guidelines for creating MATLAB code that will be easy to understand, enhance, and maintain"\\
  \url{http://www.mathworks.com/matlabcentral/fileexchange/46056-matlab-style-guidelines-2-0}
\item ``The elements of MATLAB style" by 	Richard K Johnson seems to be Previous edition available as print book from Five College libraries. \url {http://umass.worldcat.org/oclc/662152769}

\end{itemize}
\end{frame}
\begin{frame}{Tips for Advanced users}
\begin{itemize}
\item Best Practices for Scientific Computing \url{http://software-carpentry.org/blog/2014/01/best-practices-has-been-published.html}
\item Use source control with GitHub\\ \url{http://blogs.mathworks.com/community/2014/10/20/matlab-and-git/}
\end{itemize}
\end{frame}
\end{document}